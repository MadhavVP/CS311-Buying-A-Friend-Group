\problemname{Title of the Problem Here}

% example use of the \illustration command
% \illustration{0.3}{filename}{Photo by \href{url_here}{link text here, e.g. author name}}

Put the problem story here. Here are some common issues to keep in mind as you
edit:
\begin{itemize}
    \item Use \LaTeX math mode for (most) numbers. The few exceptions are for
        years (e.g. 2019) or whenever a number is part of a string token.
    \item For numbers $1\,000$ or larger, use thousands separatator \verb+\\,+.
        So, for example, one million would be $1\,000\,000$.
    \item Write in active voice (i.e. avoid passive voice).
    \item Write in and present tense (avoid the word `will', especially in the
        input description). So: we prefer the text `Input starts with an
        integer $n$ followed by $n$ lines...' rather than `Input {\em will}
        start with an integer $n$, then {\em will} come $n$ lines...'.
\end{itemize}


\section*{Input}

Put the input format (syntax) description here, including constraints on all
input values. That includes number ranges, the maximum number of digits after the
decimal point for real values, the character sets and length ranges for
strings, etc. If possible, always give a count of the number of items that
follow (test cases, e.g.) rather than delimiting with a token or EOF.  The
input format should be fairly precise, so that the reader knows exactly what to
expect (it should usually be easy to write a program to read the input using
line-oriented or token-oriented parsing).

\section*{Output}

Describe the output. Kattis output judging is not sensitive to case changes or
space changes (including line formatting and blank lines). Try to maintain this
flexibility with your problem -- do not be overly precise or prescriptive here
in the output formatting description.  However, if exact textual matching or
floating-point precision is required, do specify the constraints (e.g. `your
answer should be correct within a relative or absolute error of $10^{-3}$').
